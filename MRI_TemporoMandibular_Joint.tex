\documentclass[leqno,10pt,twocolumn,a4paper]{article}
\usepackage[utf8]{inputenc}
\usepackage{graphicx}
\usepackage{xcolor}
\usepackage{amsmath}
\usepackage{units}
\graphicspath{ {./images4TheBook} }
%\usepackage{showframe}
\usepackage{lipsum}
\usepackage{fancyhdr}
\pagestyle{fancy}
\begin{document}
	\begin{titlepage}
		\begin{center}
			\vspace*{0.5cm}
			\textbf{Tiziana Robba\\ Carlotta Tanteri\\ Giulia Tanteri}\\
			\vspace*{6cm}
			\LARGE{MRI of the Temporomandibular Joint}\\
			\vspace*{0.5cm}
			\small{\textit{Correlazione fra Imaging e Patologia}}\\
			\vfill
			\includegraphics[width=4cm]{springerLogo}
		\end{center}
	\end{titlepage}
	\onecolumn
	\thispagestyle{plain}
	\pagenumbering{roman}
	\tableofcontents
	\newpage
	\thispagestyle{plain}
	\begin{flushleft}
		\vspace*{3cm}\hspace{3cm}\parbox[center]{10cm}{Ai nostri mentori, che ci hanno indicato la via che ci ha portato verso la parte migliore di noi stessi\\ Alle nostre famiglie, mariti e ai nostri bambini Francesco, Chiara, Benedetta e Greogorio Oceano\\
									Ai nostri pazienti, per aver riposto fiducia in noi e per averci dato la possibilità di imparare cose nuove ogni giorno\\ Ai nostri amici e colleghi\\ A noi stessi, poichè condividere così tanto ci ha fatto fare 
									tesoro l'uno dell'altro in un modo che non ci aspettavamo\\ \\ \textit{-- Chi vedrebbe più lontano fra un nano ed un gigante? Sicuramente il gigante poichè ha gli occhi situati ad un livello più alto
									in confronto a quelli del nano. Mettiamo il caso che il nano sia posizionato sulle spalle del gigante, chi vede più lontano? ... Così anche noi siamo il nano a cavalcioni sulle spalle di giganti. Impariamo
									dalle loro vedute e riusciamo ad andare oltre. Grazie al loro sapere cresciamo con consapevolezza e siamo capaci di dire quello che abbiamo da dire, ma non perchè siamo più grandi di loro. --
									\begin{flushright}
										\textit{Isaiah di Trani}
									\end{flushright}}}\\
	\end{flushleft}
	
	\newpage
	
	\thispagestyle{plain}
	
	\addcontentsline{toc}{section}{Prefazione di Ambra Michelotti}
	\section*{Prefazione di Ambra Michelotti}
	
	\vspace*{0.5cm}
	Questo libro editato da Tiziana Robba, Carlotta Tanteri e Giulia Tanteri ha il grande merito, il quale non è scontato, di riuscire a colmare uno spazio nel materiale informativo in ambito medico-odontoiatrico. La crescente ricerca in termini diagnostici ed approcci
	terapeutici riguardo le patologie dell'articolazione temporomandibolare, combinata con l'incessante progresso tecnologico compiuto nel campo di imaging diagnostico, ha di fatto contribuito all'aumentare il divario che vi è fra esperti di discipline differenti
	richiamati a definire il percorso diagnostico e terapeutico. Il fine di scrivere un testo multidisciplinario, secondo la prospettiva di un dentista, un radiologo ed un chirurgo maxillofacciale, ci permette di colmare questo divario attraverso l'unione delle
	competenze degli autori nei diversi campi. Questo libro è provvisto degli elementi basilari necessari per il radiologo per capire le differenti patologie dei tessuti articolari ed i cambiamenti morfologici dovuti ad esse e, allo stesso tempo, aiuta il clinico ad
	interpretare le immagini originarie da strumenti di diagnostica.\\
	Essenzialmente, gli autori hanno portato a termine un eccellente ed apprezzabile lavoro di coordinazione interdisciplinare che permetterà a chiunque leggerà e studierà questo testo facilmente accessibile di fronteggiare le complesse patologie
	temporomandibolari con esperienza maggiore.
	\begin{flushright}
		Ambra Michelotti\\
		\textit{\footnotesize{Associate Professor Department of Orthodontics and Gnatolgy University of Naples Federico II Naples, Italy}}
	\end{flushright}
	
	\newpage
	
	\thispagestyle{plain}
	
	\addcontentsline{toc}{section}{Prefazione di Daniele Regge}
	\section*{Prefazione di Daniele Regge}
	
	\vspace*{0.5cm}
	In questo volume, dott.ssa Tiziana Robba, dott.ssa Carlotta Tanteri e dott.ssa Giulia Tanteri forniscono una panoramica completa sui disturbi dell'articolazione temporomandibolare (ATM) ed il ruolo della risonanza magnetica (MRI) nelle diagnosi di esse.
	Tutti i capitoli del seguente testo seguono la stessa struttura: dopo una breve introduzione sull'epidemiologia ed eziopatogenesi, gli autori approfondiscono i concetti adottando un approccio multidisciplinare. All'inizio è data una descrizione dal punto di vista
	fisiologico seguita da una precisa panoramica delle risultanze patologiche, entrambe viste da una prospaaettiva radiologica e patologica. La comprensibilità di questo testo è data sopratutto dalla qualità delle immagini radiologiche. Vengono discusse con 
	molta rilevanza le implicazioni cliniche e terapeutiche dei risultati radiologici.\par Trovo questo testo di estrema ispirazione e informativo e lo consiglio non solo ad esperti di imaging ma anche a dentisti, gnatologi e chirurghi maxillofacciali. Sono convinto
	che questa ricerca possa aggiungere molto alla conoscenza attraverso una precisa e adeguata descrizione dei risultati della MRI e permetta dei piani accurati di trattamento nei pazienti affetti da patologie dell'ATM. Spero che ottenga il successo
	che merita.
	\begin{flushright}
		Daniele Regge\\
		\textit{\footnotesize{Radiology Unit IRCCS Candiolo Cancer Istitute University of Turin, Turin Italy}}
	\end{flushright}
	
	\newpage
	
	\thispagestyle{plain}
	
	\addcontentsline{toc}{section}{Prefazione di Florencio Monje}
	\section*{Prefazione di Florencio Monje}
	
	\vspace*{0.5cm}
	È un grande onore e privilegio collaborare per questo testo scrivendo la prefazione. I redattori hanno formato un team stimolante tramite una combinazione di specialità differenti, concentrate nella diagnosi e trattamento di queste patologie, come
	radiologia, odontoiatria e chirurgia maxillofacciale.\par Certamente le autrici hanno fatto un tour della MRI senza ignorare altre modalità di imaging dell'ATM. I dati sull'anatomia dell'ATM apllicati alla MRI, così come le dinamiche dell'articolazione,
	ci dà le basi fondamentali per le seguenti sezioni del testo. La chiave per la corretta interpretazione del materiale di imaging dell'ATM si trova nella scrupolosa conoscenza dell'anatomia ed una comprensione delle funzioni e disfunzioni delle ATM.
	In altre parole, i professionisti del campo delle patologie di questa articolazione dovrebbero avere abbastanza esperienza sui materiali di imaging di quest'area.\par Il processo diagnostico è specialmente importante in quanto una diagnosi 
	incorretta è la causa più frequente di un trattamento fallace. In ogni caso, il materiale radiologico non dev'essere mai consultato in modo isolato. La scelta di qualunque trattamento deve basarsi in primo luogo sulla combinazione fra dati clinici
	e radiologici assieme ad altri fattori come l'impatto della malattia sul paziente e di una prognosi nel caso non vi sia alcun trattamento. Da chirurgo maxillofacciale, e da persona affascinata dalle patologie dell'ATM, apprezzo veramente questo libro.
	È diventato uno strumento lavorativo nella diagnosi dell'ATM, indipendentemente dall'approccio.
	\begin{flushright}
		Florencio Monje\\
		\textit{\footnotesize{Department of Oral and Maxillofacial Surgery University Hospital Infanta Cristina Badajoz, Spain}}
	\end{flushright}
	
	\newpage
	
	\thispagestyle{plain}
	
	\addcontentsline{toc}{section}{Prefazione}
	\section*{Prefazione}
	
	\vspace*{0.5cm}
	Questo libro è il risultato di una cooperazione duratura fra un radiologo, un chirurgo maxillofacciale ed un dentista che si è specializzato nelle patologie dell'articolazione temporomandibolare nel corso degli anni. È grazie al nostro costante scambio
	di idee, competenze e considerazioni che è stato necessario lo sviluppo di questo testo. L'urgenza di condividere un linguaggio comune ed il bisogno costante di affidarsi alla reciproca competenza ci ha fatto realizzare di voler unire le sponde, 
	prima di tutto fra noi. \par Lo scopo di questo libro sta nel fornire, in un modo facile da consultare, una conoscenza essenziale per affrontare l'interpretazione della risonanza magnetica dell'ATM. Il focus del nostro lavoro è stato nel presentare
	le condizioni dell'ATM a gnatologi e professionisti odontoiatri che incontrano nella loro pratica quotidiana. Il clinico infatti, spesso non ha familiarità sull'aspetto tecnico radiologico con gli esami che prescrivono e di rado sanno analizzare le 
	immagini che ricevono. Per questo motivo, abbiamo dato parecchio spazio alla descrizione dell'anatomia della MRI dell'ATM, alle correlazioni radiologiche con i disturbi più comuni e alle considerzioni tecniche che si trovano dietro le immagini in 
	alta definizione della risonanza magnetica. Allo stesso tempo, abbiamo voluto che i radiologi comprendano la rappresentazione clinica che precede la richiesta dello svolgimento di un esame radiologico per l'ATM, così come permette a loro di
	apprezzare meglio quali dettagli debbano essere presi in considerazioni durante l'indagine e valutazione delle immagini. \par La collaborazione fra le nostre tre discipline ci ha portato alla comprensione delle difficoltà degli altri reciprocamente 
	e ci ha permesso di includere materiale informativo tale da rendere questo testo prezioso per tutti i colleghi che trattano pazienti con patologie del temporomandibolare. Le illustrazioni e la terminologia chiara mira alla semplificazione dei 
	concetti i quali creano di solito confusione. \par I clinici si occupano dei segni e dei sintomi, ma a volte si affidano totalmente alla sicurezza clinica soltanto. Il nostro approccio alle patologie dell'ATM dovrebbe essere consequenziale, basato 
	su criteri diagnostici e supportato da analisi strumentali. La risonanza magnetica mostra cosa si cela dietro segni e sintomi, in supporto a scelte diagnostiche e terapeutiche. \par Speriamo sinceramente che il nostro lavoro aiuterà i radiologi
	ad una migliore assimilazione dei ragionamenti clinici dietro l'imaging dell'ATM e che aiuterà i clinici a realizzare come la MRI sta diventando una parte fondamentale della pratica di ognuno.
	\begin{flushright}
		Tiziana Robba\\ Carlotta Tanteri \\ Giulia Tanteri
	\end{flushright}
	
	\newpage
	\thispagestyle{plain}
	
	\addcontentsline{toc}{section}{Ringraziamenti}
	\section*{Ringraziamenti}
	
	\vspace*{0.5cm}
	Vorremmo ringraziare gli autori che hanno contribuito a questo libro per la loro dedizione e sforzo per il materiale di alta qualità che hanno fornito.\par Grazie a Gino Carnazza e il Dr. Eugenio Tanteri, i nostri mentori. Grazie per averci dato
	la possibilità di scrivere questo testo. Avete altruisticamente mosso un passo indietro e ci avete permesso di fare ciò per conto nostro. Grazie per averci inseganto, ispirato e condiviso il vostro sapere e grazie per il costante supporto.
	\par Grazie Prof. Gregor Slavicek per averci dato la possibilità di imparare, studiare e crescere con te. \par Grazie a tutti i colleghi che hanno contribuito con illustrazioni e immagini radiologiche. Grazie Dr. Angelo Bracco e al Dr. Nicolò
	Margolo per i notevoli disegni anatomici. \par Vorremo ringraziare le nostre madri e padri per averci dato la necessaria tranquillità per attraversare i momenti difficili e per averci sempre coperto le spalle. \par Grazie ai nostri mariti e 
	figli poichè siamo state benedette dalla vostra presenza nelle nostre vite. Grazie dal profondo dei nostri cuori per il vostro supporto e per la vostra pazienza. \par Infine Grazie ai dottori, tecnici, staff, infermieri e buoni amici che lavorano
	con noi giornalmente. Grazie per il costante scambio di conoscenze, interrogandoci e trascorrendo le vostre vite con noi.
	
	\newpage
	
	\pagenumbering{arabic}
	\twocolumn
	\fancyhead[L]{1 Risonanza magnetica dell'ATM: Considerazioni tecniche}
	\fancyhead[R]{V. Clementi e T. Robba}
	\begin{flushleft}
		\section{Risonanza Magnetica dell'ATM: Considerazioni Tecniche}
	\end{flushleft}
	\textbf{Punti chiave}\\
	\begin{itemize}
		\item La risonanza magnetica è una tecnica di imaging multiparametrica basata sull'assorbimento di energia da parte dei nuclei atomici dei tessuti ed il successivo ritorno del sistema al suo stato iniziale. Affinchè avvenga, il paziente
		dev'essere inserito in dei campi magnetici appositamente generati e vengono utilizzate della radiazioni elettromagnetiche non-ionizzanti.
		\item I principali parametri di contrasto utilizzatio per la generazione di immagini sono: densità protonica, $T_1$ e $T_2$. Questi ultimi due sono parametri intrinseci di qualunque tessuto, relativi alla loro struttura microscopica, i 
		quali influenzano il modo in cui il sistema ritorna al suo equilibrio dopo l'assorbimento di energia a radiofrequenza (RF). L'incrocio di parametri può disporre un'ampia variazione di contrasto fra tessuti, ed è scelto in base al contesto 
		clinico.
		\item Le immagini sono generate tramite una succesioni di impulsi di radiofrequenza e campi magnetici mutevoli. Vi sono due tipi di sequenze: spin echo e gradient echo. Le sequenze spin echo sono le più utilizzate poichè forniscono
		dei dettagli anatomici di qualità, grazie al loro SNR. Le sequenze gradient echo sono utilizzate per diminuire i tempi di acquisizione, sono più sensibili ai cambiamenti della predisposizione magnetica dei tessuti, e possono fornire 
		informazioni riguardo depositi come quelli di Calcio o emosiderina.
		\item La MRI è una tecnica di imaging tomografica: rappresenta volumi di regioni anatomiche attraverso immagini 2D. La MR, a differenza di altre tecniche di imaging , può direttamente acquisire lungo piani obliqui. Acquisizioni
		3D sono inoltre possibili, permettendo una costruzione 3D isotropica del volume.
		\item Nel corso degli anni, sistemi clinici a risonanza magnetica hanno offerto un crescente numero di sequenze e tecniche, molte delle quali sono pensate per ridurre i tempi acquisizione attraverso modalità differenti di 
		acquisizione dei dati (tecniche ad acquisizione veloce) e della soppressione del fat-signal (tecniche di soppressione fat-signal).
		\item La MR non usa radiazioni ionizzanti è può perciò essere considerata una tecnica a basso rischio. Tuttavia, la MR può lo stesso comportare dei rischi per operatori e pazienti, e possono essere limitati attraverso aree designate
		e procedura di sicurezza applicate nella pratica quotidiana.
	\end{itemize}
	\subsection{Introduzione}
	L'imaging dell'articolazione temporomandibolare (ATM) è stato in costante evoluzione insieme all'avanzamento delle tecnologie di imaging. Nonostante molte modalità di imaging siano attualmente in uso per l'esaminazione dell'ATM (per
	esempio tomografia computerizzata cone beam --CBCT-- e tomografia computerizata multidetector --MDCT--), l'uso dell'imaging a risonanza magnetica è aumentato grazie alla sua grande risoluzione di contrasto, la sua forza 
	nell'evidenziare strutture di tessuto molle e segni di infiammazione e la sua capacità di acquisizione di imaging dinamico per dimostrare la funzionalità dell'articolazione (Bag et al. 2014). Oltretutto, la MRI non comporta l'utilizzo di radiazioni
	ionizzanti e questo aiuta nella limitazione all'esposizione del paziente (Niraj et al 2016). D'altro canto, i relativi svantaggi della MRI in confronto alla CT includono tecnica di scansione più complessa e dei tempi di acquisizione piu lunghi.
	I vantaggi della CT in confronto alla MRI sono: dettagli ossei migliori ed una valutazione 3D di patologie dello sviluppo, congenite e traumatiche (Bag et al 2014; Niraj et al. 2016). In questo capitolo, il lettore otterrà le informazioni
	essenziali per comprendere i principi fisici che stanno alla base della creazione di immagini MR.
	\subsection{Principi fisici dell'imaging a risonanza magnetica}
	\subsubsection{Nucleo e spin} 	
	La tecnica di imaging MR è basata sull'assorbimento di energia dai nuclei atomici seguito da un ritorno del sistema al suo stato iniziale. In particolare, la maggior parte della applicazioni cliniche della MR sono 
	basate su nuclei di idrogeno, infatti, esso è uno degli elementi più comuni in natura e molto
	abbondante nel corpo umano.\par Gli atomi sono caratterizzati da tre principali particelle: protoni, che hanno una carica positiva, neutroni, che non hanno carica ed elettroni, i quali hanno una carica negativa. \par Il nucleo
	atomico è caricato positivamente a causa della presenza dei protoni e dei neutroni. Gli elettroni sono sparsi in orbitali che circondano il nucleo. Ogni elemento è definito tipicamente da un numero di protoni ed elettroni,
	mentre il nucleo può contenere un numero variabile di neutroni caratterizzando in tal modo isotopi differenti dello stesso elemento. Il nucleo di idrogeno contiene un singolo protone e nessun neutrone (Fig. 1.1).
	\par La completa comprensione del fenomeno della MR è basato sulla teoria della meccanica quantistica, il modello più completo per l'interpretazione del mondo microscopico. Tuttavia, la meccanica quantistica è di solito
	molto lontana dalla nostra interpretazione intuitiva, che è basata sull'esperienza macroscopica del mondo e descritta attraverso modelli fisici classici. Per questa ragione, è comune utilizzare dei concetti di meccanica
	quantistica, insieme a modelli meccanici classici, per spiegare e capire meglio certi aspetti della MR. \par Il nucleo atomico ha la proprietà intrinseca della rotazione sul proprio asse, come una trottola. La quantità fisica
	che descrive questa particolarità è un vettore chiamato \textit{spin momento angolare}, anche chiamato \textit{spin} (Fig. 1.2a). Dalla fisica dell'elettromagnetismo, è anche noto che una carica in movimento crea un 
	campo magnetico. La fonte di questo campo magnetico può essere rappresentata in fisica attraverso un dipolo, esso può essere immaginato come un piccolo magnete, con un polo nord ed un polo sud. Quindi un nucleo
	può essere rappresentato come una trottola data la sua rotazione attorno al proprio asse, così come un piccolo magnete dato il suo piccolo campo magnetico (Fig. 1.2b). I piccoli dipoli, sono all'origine del segnale MR.
	\par Basandoci su quello che è appena stato descritto, un volume di tessuto di un paziente può essere immaginato come un gruppo di piccoli magneti, tutti nuclei di idrogeno (o spins), ognuno di loro ruota attorno al proprio
	asse generando una microscopica calamita. In condizioni normali, al di fuori di un forte campo magnetico, ogni orientamento dello spin è possibile, i campi magnetici generati dal nucleo si annullano a vicenda e l'effetto
	complessivoè che non vi è sorta di magnetizzazione sul tessuto (Fig. 1.3).
	\subsubsection{Il Fenomeno della Risonanza e la Precessione di Larmor}
	Quando il paziente viene introdotto dentro una divisa e ad un costante campo magnetico ad alta intensità, appositamente creato all'interno dello scanner MR, chiamato $\boldsymbol{B_0}$, l'idrogeno ruota, il quale era
	orientato in modo casuale fino a quel punto, si muove per allinearsi lungo il principale campo magnetico $\boldsymbol{B_0}$ in un orientamento parallelo o anti-parallelo (Fig. 1.4). L'allineamento anti-parallelo comporta 
	un'energia superiore a differenza di quello parallelo e quest'ultimo è più frequente. Nei modelli di fisica classica, se una trottola viene spostata dal proprio asse mentre ci gira, comincerà pure a ruotare attorno la direzione della gravità. 
	Di nuovo, lo spin può essere considerato come una trottola, e quando lo spin è esposto ad un campo magnetico $\boldsymbol{B_0}$ si comporta come una trottola sul campo gravitazionale ed incomincia a ruotare attorno
	l'asse di $\boldsymbol{B_0}$ con moto caratteristico noto come \textit{precessione} (Fig. 1.5). \par L'equazione di Larmor descrive la relazione fra l'intensità del campo magnetico $\boldsymbol{B_0}$ e la frequenza di 
	rotazione della precessione di spin:
	\begin{equation}
		\omega_0=\gamma B_0
	\end{equation}
	dove $\omega_0$ è nota come la frequenza della precessione di Larmor o frequenza di risonanza ed è espressa in MHz, $\gamma$ è il rapporto giromagnetico (unico per ciascun atomo), espresso in MHz/T e $B_0$ è 
	l'intensità del campo magnetico misurato in Tesla. \par I protoni hanno un rapport giromagnetico di 42.58 MHz/T, e la corrispondente frequenza di Larmor a 1.5 T è 63.87 MHz. Questo valore può essere approssimato
	intorno ad 1 KHz, il quale è la frequenza Larmor corrispondente all'intensità del campo magnetico della terra. \par Come conseguenza di quanto appena descritto, ogni qualvolta che un paziente è portato all'interno di un
	campo magnetico esterno, l'effetto complessivo è l'apparizione di una magnetizzazione microscopica che può essere rappresentata da un vettore \textbf{M}, con la stessa direzione e verso del campo magnetico esterno
	$\boldsymbol{B_0}$ (Fig. 1.6). Non vi è nessun vettore di magnetizzazione quando un tessuto non è posizionato in un campo magnetico esterno. \par Da un punto di vista meccanico-quantistico, quando la componente 
	spin è misurata lungo l'asse un asse (l'asse $z$ per esempio) è possibile ottenere solamente un numero finito di valori (valori quantizzati), relativi ad un numero che descrive il momento angolare dello spin: il numero 
	quantico di spin $I$. Questo numero differisce da nucleo a nucleo. Nuclei con un numero quantico di spin $I=0$ non possono essere utilizzati per la MR. Il nucleo di idrogeno ha il numero quantico di spin $I=\nicefrac{1}{2}$,
	il quale lo rende adatto per la creazione di un segnale MR. Altri nuclei, per esempio Carbonio-13, Azoto-14, Fluoro-19, Fosforo-31 e Sodio-23 sono caratterizzati da un $I\neq 0$ e possono potenzialmente generare 
	segnale MR. Tuttavia, questi elementi sono meno abbondanti nei tessuti biologici in confronto all'idrogeno; Pertanto il loro uso nella MRI non è uno standard ed è limitato solo a specifiche applicazioni di ricerca.
	\par Al di fuori del campo $\boldsymbol{B_0}$, quando si misura il componente dello spin del nucleo lungo l'asse $z$, gli unici posiibili risultati sono due valori, corrispondenti agli stati dello spin UP e DOWN, entrambi
	avendo la stessa energia. Il numero di spin UP è equivalente al numero degli spin DOWN e non vi è alcun effetto macroscopico. Al contrario, quando un paziente viene inserito nel campo magnetico principale $\boldsymbol{B_0}$,
	le due possibili configurazioni dello spin dell'idrogeno --rispetto all'asse di $\boldsymbol{B_0}$, UP (anti-parallelo) o DOWN (parallelo)-- diventano energicamente differenti, con l'allineamento UP corrispondente allo stato
	energetico superiore (Fig. 1.7). La distribuzione dei vari spin, fra stati d'energia UP e DOWN, non è più uguale ed è collegato al movimento termico microscopico all'interno del tessuto ed alla intensità del campo magnetico 
	esterno $\boldsymbol{B_0}$. A temperatura corporea e con il tipico campo magnetico esterno usato nella clinica pratica  
	
	
	




















\end{document}