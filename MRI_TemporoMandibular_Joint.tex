\documentclass[10pt,twocolumn,a4paper]{article}
\usepackage[utf8]{inputenc}
\usepackage{graphicx}
\graphicspath{ {./images4TheBook} }
%\usepackage{showframe}
\usepackage{lipsum}
\usepackage{fancyhdr}
\begin{document}
	\begin{titlepage}
		\begin{center}
			\vspace*{0.5cm}
			\textbf{Tiziana Robba\\ Carlotta Tanteri\\ Giulia Tanteri}\\
			\vspace*{6cm}
			\LARGE{MRI of the Temporomandibular Joint}\\
			\vspace*{0.5cm}
			\small{\textit{Correlazione fra Imaging e Patologia}}\\
			\vfill
			\includegraphics[width=4cm]{springerLogo}
		\end{center}
	\end{titlepage}
	\onecolumn
	\begin{flushleft}
		\vspace*{3cm}\hspace{3cm}\parbox[center]{10cm}{Ai nostri mentori, che ci hanno indicato la via che ci ha portato verso la parte migliore di noi stessi\\ Alle nostre famiglie, mariti e ai nostri bambini Francesco, Chiara, Benedetta e Greogorio Oceano\\
									Ai nostri pazienti, per aver riposto fiducia in noi e per averci dato la possibilità di imparare cose nuove ogni giorno\\ Ai nostri amici e colleghi\\ A noi stessi, poichè condividere così tanto ci ha fatto fare 
									tesoro l'uno dell'altro in un modo che non ci aspettavamo\\ \\ \textit{-- Chi vedrebbe più lontano fra un nano ed un gigante? Sicuramente il gigante poichè ha gli occhi situati ad un livello più alto
									in confronto a quelli del nano. Mettiamo il caso che il nano sia posizionato sulle spalle del gigante, chi vede più lontano? ... Così anche noi siamo il nano a cavalcioni sulle spalle di giganti. Impariamo
									dalle loro vedute e riusciamo ad andare oltre. Grazie al loro sapere cresciamo con consapevolezza e siamo capaci di dire quello che abbiamo da dire, ma non perchè siamo più grandi di loro. --
									\begin{flushright}
										\textit{Isaiah di Trani}
									\end{flushright}}}\\
	\end{flushleft}
	
	\newpage
	
	\textbf{\Large{Prefazione di Ambra Michelotti}}\\
	
	\vspace*{0.5cm}
	Questo libro editato da Tiziana Robba, Carlotta Tanteri e Giulia Tanteri ha il grande merito, il quale non è scontato, di riuscire a colmare uno spazio nel materiale informativo in ambito medico-odontoiatrico. La crescente ricerca in termini diagnostici ed approcci
	terapeutici riguardo le patologie dell'articolazione temporomandibolare, combinata con l'incessante progresso tecnologico compiuto nel campo di imaging diagnostico, ha di fatto contribuito all'aumentare il divario che vi è fra esperti di discipline differenti
	richiamati a definire il percorso diagnostico e terapeutico. Il fine di scrivere un testo multidisciplinario, secondo la prospettiva di un dentista, un radiologo ed un chirurgo maxillofacciale, ci permette di colmare questo divario attraverso l'unione delle
	competenze degli autori nei diversi campi. Questo libro è provvisto degli elementi basilari necessari per il radiologo per capire le differenti patologie dei tessuti articolari ed i cambiamenti morfologici dovuti ad esse e, allo stesso tempo, aiuta il clinico ad
	interpretare le immagini originarie da strumenti di diagnostica.\\
	Essenzialmente, gli autori hanno portato a termine un eccellente ed apprezzabile lavoro di coordinazione interdisciplinare che permetterà a chiunque leggerà e studierà questo testo facilmente accessibile di fronteggiare le complesse patologie
	temporomandibolari con esperienza maggiore.
	\begin{flushright}
		Ambra Michelotti\\
		\textit{\footnotesize{Associate Professor Department of Orthodontics and Gnatolgy University of Naples Federico II Naples, Italy}}
	\end{flushright}
	
	\newpage
	
	\textbf{\Large{Prefazione di Daniele Regge}}\\
	
	\vspace*{0.5cm}
	In questo volume, dott.ssa Tiziana Robba, dott.ssa Carlotta Tanteri e dott.ssa Giulia Tanteri forniscono una panoramica completa sui disturbi dell'articolazione temporomandibolare (ATM) ed il ruolo della risonanza magnetica (MRI) nelle diagnosi di esse.
	Tutti i capitoli del seguente testo seguono la stessa struttura: dopo una breve introduzione sull'epidemiologia ed eziopatogenesi, gli autori approfondiscono i concetti adottando un approccio multidisciplinare. All'inizio è data una descrizione dal punto di vista
	fisiologico seguita da una precisa panoramica delle risultanze patologiche, entrambe viste da una prospaaettiva radiologica e patologica. La comprensibilità di questo testo è data sopratutto dalla qualità delle immagini radiologiche. Vengono discusse con 
	molta rilevanza le implicazioni cliniche e terapeutiche dei risultati radiologici.\par Trovo questo testo di estrema ispirazione e informativo e lo consiglio non solo ad esperti di imaging ma anche a dentisti, gnatologi e chirurghi maxillofacciali. Sono convinto
	che questa ricerca possa aggiungere molto alla conoscenza attraverso una precisa e adeguata descrizione dei risultati della MRI e permetta dei piani accurati di trattamento nei pazienti affetti da patologie dell'ATM. Spero che ottenga il successo
	che merita.
	\begin{flushright}
		Daniele Regge\\
		\textit{\footnotesize{Radiology Unit IRCCS Candiolo Cancer Istitute University of Turin, Turin Italy}}
	\end{flushright}
	
	\newpage
	
	\textbf{\Large{Prefazione di Florencio Monje}}\\
	
	\vspace*{0.5cm}
	È un grande onore e privilegio collaborare per questo testo scrivendo la prefazione. I redattori hanno formato un team stimolante tramite una combinazione di specialità differenti, concentrate nella diagnosi e trattamento di queste patologie, come
	radiologia, odontoiatria e chirurgia maxillofacciale.\par Certamente le autrici hanno fatto un tour della MRI senza ignorare altre modalità di imaging dell'ATM. I dati sull'anatomia dell'ATM apllicati alla MRI, così come le dinamiche dell'articolazione,
	ci dà le basi fondamentali per le seguenti sezioni del testo. La chiave per la corretta interpretazione del materiale di imaging dell'ATM si trova nella scrupolosa conoscenza dell'anatomia ed una comprensione delle funzioni e disfunzioni delle ATM.
	In altre parole, i professionisti del campo delle patologie di questa articolazione dovrebbero avere abbastanza esperienza sui materiali di imaging di quest'area.\par Il processo diagnostico è specialmente importante in quanto una diagnosi 
	incorretta è la causa più frequente di un trattamento fallace. In ogni caso, il materiale radiologico non dev'essere mai consultato in modo isolato. La scelta di qualunque trattamento deve basarsi in primo luogo sulla combinazione fra dati clinici
	e radiologici assieme ad altri fattori come l'impatto della malattia sul paziente e di una prognosi nel caso non vi sia alcun trattamento. Da chirurgo maxillofacciale, e da persona affascinata dalle patologie dell'ATM, apprezzo veramente questo libro.
	È diventato uno strumento lavorativo nella diagnosi dell'ATM, indipendentemente dall'approccio.
	\begin{flushright}
		Florencio Monje\\
		\textit{\footnotesize{Department of Oral and Maxillofacial Surgery University Hospital Infanta Cristina Badajoz, Spain}}
	\end{flushright}
	
	\newpage
	
	\textbf{\Large{Prefazione}}\\
	
	\vspace*{0.5cm}
	Questo libro è il risultato di una cooperazione duratura fra un radiologo, un chirurgo maxillofacciale ed un dentista che si è specializzato nelle patologie dell'articolazione temporomandibolare nel corso degli anni. È grazie al nostro costante scambio
	di idee, competenze e considerazioni che è stato necessario lo sviluppo di questo testo. L'urgenza di condividere un linguaggio comune ed il bisogno costante di affidarsi alla reciproca competenza ci ha fatto realizzare di voler unire le sponde, 
	prima di tutto fra noi. \par Lo scopo di questo libro sta nel fornire, in un modo facile da consultare, una conoscenza essenziale per affrontare l'interpretazione della risonanza magnetica dell'ATM. Il focus del nostro lavoro è stato nel presentare
	le condizioni dell'ATM a gnatologi e professionisti odontoiatri incontrano nel loro quotidiano
	
	





















\end{document}