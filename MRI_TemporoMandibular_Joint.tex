\documentclass[leqno,10pt,twocolumn,a4paper]{article}
\usepackage[utf8]{inputenc}
\usepackage{graphicx}
\usepackage{xcolor}
\usepackage{amsmath}
\usepackage{units}
\graphicspath{ {./images4TheBook} }
%\usepackage{showframe}
\usepackage{lipsum}
\usepackage{fancyhdr}
\pagestyle{fancy}
\begin{document}
	\begin{titlepage}
		\begin{center}
			\vspace*{0.5cm}
			\textbf{Tiziana Robba\\ Carlotta Tanteri\\ Giulia Tanteri}\\
			\vspace*{6cm}
			\LARGE{MRI of the Temporomandibular Joint}\\
			\vspace*{0.5cm}
			\small{\textit{Correlazione fra Imaging e Patologia}}\\
			\vfill
			\includegraphics[width=4cm]{springerLogo}
		\end{center}
	\end{titlepage}
	\onecolumn
	\thispagestyle{plain}
	\pagenumbering{roman}
	\tableofcontents
	\newpage
	\thispagestyle{plain}
	\begin{flushleft}
		\vspace*{3cm}\hspace{3cm}\parbox[center]{10cm}{Ai nostri mentori, che ci hanno indicato la via che ci ha portato verso la parte migliore di noi stessi\\ Alle nostre famiglie, mariti e ai nostri bambini Francesco, Chiara, Benedetta e Greogorio Oceano\\
									Ai nostri pazienti, per aver riposto fiducia in noi e per averci dato la possibilità di imparare cose nuove ogni giorno\\ Ai nostri amici e colleghi\\ A noi stessi, poichè condividere così tanto ci ha fatto fare 
									tesoro l'uno dell'altro in un modo che non ci aspettavamo\\ \\ \textit{-- Chi vedrebbe più lontano fra un nano ed un gigante? Sicuramente il gigante poichè ha gli occhi situati ad un livello più alto
									in confronto a quelli del nano. Mettiamo il caso che il nano sia posizionato sulle spalle del gigante, chi vede più lontano? ... Così anche noi siamo il nano a cavalcioni sulle spalle di giganti. Impariamo
									dalle loro vedute e riusciamo ad andare oltre. Grazie al loro sapere cresciamo con consapevolezza e siamo capaci di dire quello che abbiamo da dire, ma non perchè siamo più grandi di loro. --
									\begin{flushright}
										\textit{Isaiah di Trani}
									\end{flushright}}}\\
	\end{flushleft}
	
	\newpage
	
	\thispagestyle{plain}
	
	\addcontentsline{toc}{section}{Prefazione di Ambra Michelotti}
	\section*{Prefazione di Ambra Michelotti}
	
	\vspace*{0.5cm}
	Questo libro editato da Tiziana Robba, Carlotta Tanteri e Giulia Tanteri ha il grande merito, il quale non è scontato, di riuscire a colmare uno spazio nel materiale informativo in ambito medico-odontoiatrico. La crescente ricerca in termini diagnostici ed approcci
	terapeutici riguardo le patologie dell'articolazione temporomandibolare, combinata con l'incessante progresso tecnologico compiuto nel campo di imaging diagnostico, ha di fatto contribuito all'aumentare il divario che vi è fra esperti di discipline differenti
	richiamati a definire il percorso diagnostico e terapeutico. Il fine di scrivere un testo multidisciplinario, secondo la prospettiva di un dentista, un radiologo ed un chirurgo maxillofacciale, ci permette di colmare questo divario attraverso l'unione delle
	competenze degli autori nei diversi campi. Questo libro è provvisto degli elementi basilari necessari per il radiologo per capire le differenti patologie dei tessuti articolari ed i cambiamenti morfologici dovuti ad esse e, allo stesso tempo, aiuta il clinico ad
	interpretare le immagini originarie da strumenti di diagnostica.\\
	Essenzialmente, gli autori hanno portato a termine un eccellente ed apprezzabile lavoro di coordinazione interdisciplinare che permetterà a chiunque leggerà e studierà questo testo facilmente accessibile di fronteggiare le complesse patologie
	temporomandibolari con esperienza maggiore.
	\begin{flushright}
		Ambra Michelotti\\
		\textit{\footnotesize{Associate Professor Department of Orthodontics and Gnatolgy University of Naples Federico II Naples, Italy}}
	\end{flushright}
	
	\newpage
	
	\thispagestyle{plain}
	
	\addcontentsline{toc}{section}{Prefazione di Daniele Regge}
	\section*{Prefazione di Daniele Regge}
	
	\vspace*{0.5cm}
	In questo volume, dott.ssa Tiziana Robba, dott.ssa Carlotta Tanteri e dott.ssa Giulia Tanteri forniscono una panoramica completa sui disturbi dell'articolazione temporomandibolare (ATM) ed il ruolo della risonanza magnetica (MRI) nelle diagnosi di esse.
	Tutti i capitoli del seguente testo seguono la stessa struttura: dopo una breve introduzione sull'epidemiologia ed eziopatogenesi, gli autori approfondiscono i concetti adottando un approccio multidisciplinare. All'inizio è data una descrizione dal punto di vista
	fisiologico seguita da una precisa panoramica delle risultanze patologiche, entrambe viste da una prospaaettiva radiologica e patologica. La comprensibilità di questo testo è data sopratutto dalla qualità delle immagini radiologiche. Vengono discusse con 
	molta rilevanza le implicazioni cliniche e terapeutiche dei risultati radiologici.\par Trovo questo testo di estrema ispirazione e informativo e lo consiglio non solo ad esperti di imaging ma anche a dentisti, gnatologi e chirurghi maxillofacciali. Sono convinto
	che questa ricerca possa aggiungere molto alla conoscenza attraverso una precisa e adeguata descrizione dei risultati della MRI e permetta dei piani accurati di trattamento nei pazienti affetti da patologie dell'ATM. Spero che ottenga il successo
	che merita.
	\begin{flushright}
		Daniele Regge\\
		\textit{\footnotesize{Radiology Unit IRCCS Candiolo Cancer Istitute University of Turin, Turin Italy}}
	\end{flushright}
	
	\newpage
	
	\thispagestyle{plain}
	
	\addcontentsline{toc}{section}{Prefazione di Florencio Monje}
	\section*{Prefazione di Florencio Monje}
	
	\vspace*{0.5cm}
	È un grande onore e privilegio collaborare per questo testo scrivendo la prefazione. I redattori hanno formato un team stimolante tramite una combinazione di specialità differenti, concentrate nella diagnosi e trattamento di queste patologie, come
	radiologia, odontoiatria e chirurgia maxillofacciale.\par Certamente le autrici hanno fatto un tour della MRI senza ignorare altre modalità di imaging dell'ATM. I dati sull'anatomia dell'ATM apllicati alla MRI, così come le dinamiche dell'articolazione,
	ci dà le basi fondamentali per le seguenti sezioni del testo. La chiave per la corretta interpretazione del materiale di imaging dell'ATM si trova nella scrupolosa conoscenza dell'anatomia ed una comprensione delle funzioni e disfunzioni delle ATM.
	In altre parole, i professionisti del campo delle patologie di questa articolazione dovrebbero avere abbastanza esperienza sui materiali di imaging di quest'area.\par Il processo diagnostico è specialmente importante in quanto una diagnosi 
	incorretta è la causa più frequente di un trattamento fallace. In ogni caso, il materiale radiologico non dev'essere mai consultato in modo isolato. La scelta di qualunque trattamento deve basarsi in primo luogo sulla combinazione fra dati clinici
	e radiologici assieme ad altri fattori come l'impatto della malattia sul paziente e di una prognosi nel caso non vi sia alcun trattamento. Da chirurgo maxillofacciale, e da persona affascinata dalle patologie dell'ATM, apprezzo veramente questo libro.
	È diventato uno strumento lavorativo nella diagnosi dell'ATM, indipendentemente dall'approccio.
	\begin{flushright}
		Florencio Monje\\
		\textit{\footnotesize{Department of Oral and Maxillofacial Surgery University Hospital Infanta Cristina Badajoz, Spain}}
	\end{flushright}
	
	\newpage
	
	\thispagestyle{plain}
	
	\addcontentsline{toc}{section}{Prefazione}
	\section*{Prefazione}
	
	\vspace*{0.5cm}
	Questo libro è il risultato di una cooperazione duratura fra un radiologo, un chirurgo maxillofacciale ed un dentista che si è specializzato nelle patologie dell'articolazione temporomandibolare nel corso degli anni. È grazie al nostro costante scambio
	di idee, competenze e considerazioni che è stato necessario lo sviluppo di questo testo. L'urgenza di condividere un linguaggio comune ed il bisogno costante di affidarsi alla reciproca competenza ci ha fatto realizzare di voler unire le sponde, 
	prima di tutto fra noi. \par Lo scopo di questo libro sta nel fornire, in un modo facile da consultare, una conoscenza essenziale per affrontare l'interpretazione della risonanza magnetica dell'ATM. Il focus del nostro lavoro è stato nel presentare
	le condizioni dell'ATM a gnatologi e professionisti odontoiatri che incontrano nella loro pratica quotidiana. Il clinico infatti, spesso non ha familiarità sull'aspetto tecnico radiologico con gli esami che prescrivono e di rado sanno analizzare le 
	immagini che ricevono. Per questo motivo, abbiamo dato parecchio spazio alla descrizione dell'anatomia della MRI dell'ATM, alle correlazioni radiologiche con i disturbi più comuni e alle considerzioni tecniche che si trovano dietro le immagini in 
	alta definizione della risonanza magnetica. Allo stesso tempo, abbiamo voluto che i radiologi comprendano la rappresentazione clinica che precede la richiesta dello svolgimento di un esame radiologico per l'ATM, così come permette a loro di
	apprezzare meglio quali dettagli debbano essere presi in considerazioni durante l'indagine e valutazione delle immagini. \par La collaborazione fra le nostre tre discipline ci ha portato alla comprensione delle difficoltà degli altri reciprocamente 
	e ci ha permesso di includere materiale informativo tale da rendere questo testo prezioso per tutti i colleghi che trattano pazienti con patologie del temporomandibolare. Le illustrazioni e la terminologia chiara mira alla semplificazione dei 
	concetti i quali creano di solito confusione. \par I clinici si occupano dei segni e dei sintomi, ma a volte si affidano totalmente alla sicurezza clinica soltanto. Il nostro approccio alle patologie dell'ATM dovrebbe essere consequenziale, basato 
	su criteri diagnostici e supportato da analisi strumentali. La risonanza magnetica mostra cosa si cela dietro segni e sintomi, in supporto a scelte diagnostiche e terapeutiche. \par Speriamo sinceramente che il nostro lavoro aiuterà i radiologi
	ad una migliore assimilazione dei ragionamenti clinici dietro l'imaging dell'ATM e che aiuterà i clinici a realizzare come la MRI sta diventando una parte fondamentale della pratica di ognuno.
	\begin{flushright}
		Tiziana Robba\\ Carlotta Tanteri \\ Giulia Tanteri
	\end{flushright}
	
	\newpage
	\thispagestyle{plain}
	
	\addcontentsline{toc}{section}{Ringraziamenti}
	\section*{Ringraziamenti}
	
	\vspace*{0.5cm}
	Vorremmo ringraziare gli autori che hanno contribuito a questo libro per la loro dedizione e sforzo per il materiale di alta qualità che hanno fornito.\par Grazie a Gino Carnazza e il Dr. Eugenio Tanteri, i nostri mentori. Grazie per averci dato
	la possibilità di scrivere questo testo. Avete altruisticamente mosso un passo indietro e ci avete permesso di fare ciò per conto nostro. Grazie per averci inseganto, ispirato e condiviso il vostro sapere e grazie per il costante supporto.
	\par Grazie Prof. Gregor Slavicek per averci dato la possibilità di imparare, studiare e crescere con te. \par Grazie a tutti i colleghi che hanno contribuito con illustrazioni e immagini radiologiche. Grazie Dr. Angelo Bracco e al Dr. Nicolò
	Margolo per i notevoli disegni anatomici. \par Vorremo ringraziare le nostre madri e padri per averci dato la necessaria tranquillità per attraversare i momenti difficili e per averci sempre coperto le spalle. \par Grazie ai nostri mariti e 
	figli poichè siamo state benedette dalla vostra presenza nelle nostre vite. Grazie dal profondo dei nostri cuori per il vostro supporto e per la vostra pazienza. \par Infine Grazie ai dottori, tecnici, staff, infermieri e buoni amici che lavorano
	con noi giornalmente. Grazie per il costante scambio di conoscenze, interrogandoci e trascorrendo le vostre vite con noi.
	
	\newpage
	
	\pagenumbering{arabic}
	\twocolumn
	\fancyhead[L]{1 Risonanza magnetica dell'ATM: Considerazioni tecniche}
	\fancyhead[R]{V. Clementi e T. Robba}
	\begin{flushleft}
		\section{Risonanza Magnetica dell'ATM: Considerazioni Tecniche}
	\end{flushleft}
	\textbf{Punti chiave}\\
	\begin{itemize}
		\item La risonanza magnetica è una tecnica di imaging multiparametrica basata sull'assorbimento di energia da parte dei nuclei atomici dei tessuti ed il successivo ritorno del sistema al suo stato iniziale. Affinchè avvenga, il paziente
		dev'essere inserito in dei campi magnetici appositamente generati e vengono utilizzate della radiazioni elettromagnetiche non-ionizzanti.
		\item I principali parametri di contrasto utilizzatio per la generazione di immagini sono: densità protonica, $T_1$ e $T_2$. Questi ultimi due sono parametri intrinseci di qualunque tessuto, relativi alla loro struttura microscopica, i 
		quali influenzano il modo in cui il sistema ritorna al suo equilibrio dopo l'assorbimento di energia a radiofrequenza (RF). L'incrocio di parametri può disporre un'ampia variazione di contrasto fra tessuti, ed è scelto in base al contesto 
		clinico.
		\item Le immagini sono generate tramite una succesioni di impulsi di radiofrequenza e campi magnetici mutevoli. Vi sono due tipi di sequenze: spin echo e gradient echo. Le sequenze spin echo sono le più utilizzate poichè forniscono
		dei dettagli anatomici di qualità, grazie al loro SNR. Le sequenze gradient echo sono utilizzate per diminuire i tempi di acquisizione, sono più sensibili ai cambiamenti della predisposizione magnetica dei tessuti, e possono fornire 
		informazioni riguardo depositi come quelli di Calcio o emosiderina.
		\item La MRI è una tecnica di imaging tomografica: rappresenta volumi di regioni anatomiche attraverso immagini 2D. La MR, a differenza di altre tecniche di imaging , può direttamente acquisire lungo piani obliqui. Acquisizioni
		3D sono inoltre possibili, permettendo una costruzione 3D isotropica del volume.
		\item Nel corso degli anni, sistemi clinici a risonanza magnetica hanno offerto un crescente numero di sequenze e tecniche, molte delle quali sono pensate per ridurre i tempi acquisizione attraverso modalità differenti di 
		acquisizione dei dati (tecniche ad acquisizione veloce) e della soppressione del fat-signal (tecniche di soppressione fat-signal).
		\item La MR non usa radiazioni ionizzanti è può perciò essere considerata una tecnica a basso rischio. Tuttavia, la MR può lo stesso comportare dei rischi per operatori e pazienti, e possono essere limitati attraverso aree designate
		e procedura di sicurezza applicate nella pratica quotidiana.
	\end{itemize}
	\subsection{Introduzione}
	L'imaging dell'articolazione temporomandibolare (ATM) è stato in costante evoluzione insieme all'avanzamento delle tecnologie di imaging. Nonostante molte modalità di imaging siano attualmente in uso per l'esaminazione dell'ATM (per
	esempio tomografia computerizzata cone beam --CBCT-- e tomografia computerizata multidetector --MDCT--), l'uso dell'imaging a risonanza magnetica è aumentato grazie alla sua grande risoluzione di contrasto, la sua forza 
	nell'evidenziare strutture di tessuto molle e segni di infiammazione e la sua capacità di acquisizione di imaging dinamico per dimostrare la funzionalità dell'articolazione (Bag et al. 2014). Oltretutto, la MRI non comporta l'utilizzo di radiazioni
	ionizzanti e questo aiuta nella limitazione all'esposizione del paziente (Niraj et al 2016). D'altro canto, i relativi svantaggi della MRI in confronto alla CT includono tecnica di scansione più complessa e dei tempi di acquisizione piu lunghi.
	I vantaggi della CT in confronto alla MRI sono: dettagli ossei migliori ed una valutazione 3D di patologie dello sviluppo, congenite e traumatiche (Bag et al 2014; Niraj et al. 2016). In questo capitolo, il lettore otterrà le informazioni
	essenziali per comprendere i principi fisici che stanno alla base della creazione di immagini MR.
	\subsection{Principi fisici dell'imaging a risonanza magnetica}
	\subsubsection{Nucleo e spin} 	
	La tecnica di imaging MR è basata sull'assorbimento di energia dai nuclei atomici seguito da un ritorno del sistema al suo stato iniziale. In particolare, la maggior parte della applicazioni cliniche della MR sono 
	basate su nuclei di idrogeno, infatti, esso è uno degli elementi più comuni in natura e molto
	abbondante nel corpo umano.\par Gli atomi sono caratterizzati da tre principali particelle: protoni, che hanno una carica positiva, neutroni, che non hanno carica ed elettroni, i quali hanno una carica negativa. \par Il nucleo
	atomico è caricato positivamente a causa della presenza dei protoni e dei neutroni. Gli elettroni sono sparsi in orbitali che circondano il nucleo. Ogni elemento è definito tipicamente da un numero di protoni ed elettroni,
	mentre il nucleo può contenere un numero variabile di neutroni caratterizzando in tal modo isotopi differenti dello stesso elemento. Il nucleo di idrogeno contiene un singolo protone e nessun neutrone (Fig. 1.1).
	\par La completa comprensione del fenomeno della MR è basato sulla teoria della meccanica quantistica, il modello più completo per l'interpretazione del mondo microscopico. Tuttavia, la meccanica quantistica è di solito
	molto lontana dalla nostra interpretazione intuitiva, che è basata sull'esperienza macroscopica del mondo e descritta attraverso modelli fisici classici. Per questa ragione, è comune utilizzare dei concetti di meccanica
	quantistica, insieme a modelli meccanici classici, per spiegare e capire meglio certi aspetti della MR. \par Il nucleo atomico ha la proprietà intrinseca della rotazione sul proprio asse, come una trottola. La quantità fisica
	che descrive questa particolarità è un vettore chiamato \textit{spin momento angolare}, anche chiamato \textit{spin} (Fig. 1.2a). Dalla fisica dell'elettromagnetismo, è anche noto che una carica in movimento crea un 
	campo magnetico. La fonte di questo campo magnetico può essere rappresentata in fisica attraverso un dipolo, esso può essere immaginato come un piccolo magnete, con un polo nord ed un polo sud. Quindi un nucleo
	può essere rappresentato come una trottola data la sua rotazione attorno al proprio asse, così come un piccolo magnete dato il suo piccolo campo magnetico (Fig. 1.2b). I piccoli dipoli, sono all'origine del segnale MR.
	\par Basandoci su quello che è appena stato descritto, un volume di tessuto di un paziente può essere immaginato come un gruppo di piccoli magneti, tutti nuclei di idrogeno (o spins), ognuno di loro ruota attorno al proprio
	asse generando una microscopica calamita. In condizioni normali, al di fuori di un forte campo magnetico, ogni orientamento dello spin è possibile, i campi magnetici generati dal nucleo si annullano a vicenda e l'effetto
	complessivoè che non vi è sorta di magnetizzazione sul tessuto (Fig. 1.3).
	\subsubsection{Il Fenomeno della Risonanza e la Precessione di Larmor}
	Quando il paziente viene introdotto dentro una divisa e ad un costante campo magnetico ad alta intensità, appositamente creato all'interno dello scanner MR, chiamato $\boldsymbol{B_0}$, l'idrogeno ruota, il quale era
	orientato in modo casuale fino a quel punto, si muove per allinearsi lungo il principale campo magnetico $\boldsymbol{B_0}$ in un orientamento parallelo o anti-parallelo (Fig. 1.4). L'allineamento anti-parallelo comporta 
	un'energia superiore a differenza di quello parallelo e quest'ultimo è più frequente. Nei modelli di fisica classica, se una trottola viene spostata dal proprio asse mentre ci gira, comincerà pure a ruotare attorno la direzione della gravità. 
	Di nuovo, lo spin può essere considerato come una trottola, e quando lo spin è esposto ad un campo magnetico $\boldsymbol{B_0}$ si comporta come una trottola sul campo gravitazionale ed incomincia a ruotare attorno
	l'asse di $\boldsymbol{B_0}$ con moto caratteristico noto come \textit{precessione} (Fig. 1.5). \par L'equazione di Larmor descrive la relazione fra l'intensità del campo magnetico $\boldsymbol{B_0}$ e la frequenza di 
	rotazione della precessione di spin:
	\begin{equation}
		\omega_0=\gamma B_0
	\end{equation}
	dove $\omega_0$ è nota come la frequenza della precessione di Larmor o frequenza di risonanza ed è espressa in MHz, $\gamma$ è il rapporto giromagnetico (unico per ciascun atomo), espresso in MHz/T e $B_0$ è 
	l'intensità del campo magnetico misurato in Tesla. \par I protoni hanno un rapport giromagnetico di 42.58 MHz/T, e la corrispondente frequenza di Larmor a 1.5 T è 63.87 MHz. Questo valore può essere approssimato
	intorno ad 1 KHz, il quale è la frequenza Larmor corrispondente all'intensità del campo magnetico della terra. \par Come conseguenza di quanto appena descritto, ogni qualvolta che un paziente è portato all'interno di un
	campo magnetico esterno, l'effetto complessivo è l'apparizione di una magnetizzazione microscopica che può essere rappresentata da un vettore \textbf{M}, con la stessa direzione e verso del campo magnetico esterno
	$\boldsymbol{B_0}$ (Fig. 1.6). Non vi è nessun vettore di magnetizzazione quando un tessuto non è posizionato in un campo magnetico esterno. \par Da un punto di vista meccanico-quantistico, quando la componente 
	spin è misurata lungo l'asse un asse (l'asse $z$ per esempio) è possibile ottenere solamente un numero finito di valori (valori quantizzati), relativi ad un numero che descrive il momento angolare dello spin: il numero 
	quantico di spin $I$. Questo numero differisce da nucleo a nucleo. Nuclei con un numero quantico di spin $I=0$ non possono essere utilizzati per la MR. Il nucleo di idrogeno ha il numero quantico di spin $I=\nicefrac{1}{2}$,
	il quale lo rende adatto per la creazione di un segnale MR. Altri nuclei, per esempio Carbonio-13, Azoto-14, Fluoro-19, Fosforo-31 e Sodio-23 sono caratterizzati da un $I\neq 0$ e possono potenzialmente generare 
	segnale MR. Tuttavia, questi elementi sono meno abbondanti nei tessuti biologici in confronto all'idrogeno; Pertanto il loro uso nella MRI non è uno standard ed è limitato solo a specifiche applicazioni di ricerca.
	\par Al di fuori del campo $\boldsymbol{B_0}$, quando si misura il componente dello spin del nucleo lungo l'asse $z$, gli unici posiibili risultati sono due valori, corrispondenti agli stati dello spin UP e DOWN, entrambi
	avendo la stessa energia. Il numero di spin UP è equivalente al numero degli spin DOWN e non vi è alcun effetto macroscopico. Al contrario, quando un paziente viene inserito nel campo magnetico principale $\boldsymbol{B_0}$,
	le due possibili configurazioni dello spin dell'idrogeno --rispetto all'asse di $\boldsymbol{B_0}$, UP (anti-parallelo) o DOWN (parallelo)-- diventano energicamente differenti, con l'allineamento UP corrispondente allo stato
	energetico superiore (Fig. 1.7). La distribuzione dei vari spin, fra stati d'energia UP e DOWN, non è più uguale ed è collegato al movimento termico microscopico all'interno del tessuto ed alla intensità del campo magnetico 
	esterno $\boldsymbol{B_0}$. A temperatura corporea e con il tipico campo magnetico esterno usato nella pratica clinica, vi è un'eccedenza (approssimativamente $10^{-6}$) di spin allo stato energetico più basso DOWN.
	Questa lieve differenza nella distribuzione del'aggregato di spin genera una magnetizzazione macroscopica \textbf{M} che è all'origine del segnale MR. Questo è il motivo per cui una delle principali sfide della tecnologia MR
	è quello di innalzare il segnale ed ottimizzare il \textit{singal-to-noise ratio} (SNR). Effettivamente più è alto l'SNR, più è l'informazione, la risoluzione spaziale ed una risoluzione temporale degli studi dinamici o con un tempo di
	scansione più basso, nelle immagini cliniche finali. %traduzione un po' del cazzo qui
	È importante sottolineare che all'aumentare di $\boldsymbol{B_0}$ aumenta anche l'intensità di \textbf{M}. Questa è la ragione per cui gli scanner MR si stanno evolvendo in dei sistemi con un $\boldsymbol{B_0}$ sempre più
	ampio. Inoltre, l'intensità della magnetizzazione macroscopica \textbf{M}, che è generata quando il paziente viene inserito nel campo esterno $\boldsymbol{B_0}$, è proporzionale al numero di protoni nel volume tissutale.
	Dal momento in cui la densità protonica (PD -- \textit{proton density}) varia fra tessuti differenti e può essere alterata dalla patolgia. La densità protonica è uno dei parametri utilizzati per generare contrasto nelle teniche di
	imaging MR.
	\subsubsection{Rilevamento del Segnale MR}
	A differenza di altre tecniche di imaging (come la CT) nella quale il costrasto è essenzialmente basato sull'attenuazione dei raggi-X, la tecnica MRI usa molteplici parametri per la generazione di contrasto. La densità protonica
	è una delle proprietà dei tessuti usate nella MR per generare contrasto, essendo la magnetizzazione macroscopica proporzionale al numero dei protoni presenti nel tessuto. Ulteriori parametri sono descritti in seguito.
	\par Per rilevare in modo efficace la magnetizzazione macroscopica, si adotta la legge di Faraday. Essa afferma che la variazione temporale del campo magnetico induce una corrente in un conduttore elettrico. La 
	magnetizzazione macroscopica \textbf{M} è inialmente orientata lungo l'asse di $\boldsymbol{B_0}$, nominato l'asse $z$, e pertanto solo la componente $M_z$ avrà un valore diverso da zero (Fig. 1.8a). Al fine di 
	acquisire il segnale, l'orientamento di \textbf{M} viene cambiato tramite un impulso di radiofrequenza (RF) trasmesso da una bobina (Fig. 1.8b, c). Se l'impulso RF è uguale a quello di Larmor, allora corrisponde alla
	frequenza di precessione degli spin intorno $\boldsymbol{B_0}$, ed il suo effetto è quello di ruotare \textbf{M} dall'asse $z$ verso il piano di $xy$ dell'angolo, chiamato angolo di flip (FA). I valori dell'angolo di flip
	possono dipendere dalla durata dell'impulso RF. Di conseguenza, la componente \textbf{M} nel piano $xy$ ($M_{xy}$) non è più nulla. Oltretutto, non appena spostatosi dall'asse di $\boldsymbol{B_0}$, \textbf{M}
	si comporta come un dipolo magnetico ed incomicia la sua precessione attorno l'asse $z$ con la frequenza Larmor $\omega_0$. Questo moto di \textbf{M} corrisponde ad un campo magnetico variabile nel tempo che
	genera un segnale elettrico nella bobina. In pratica, per motivi tecnici, è la componente $M_{xy}$ di \textbf{M} ad essere in realtà rilevata dalla bobina di ricezione, quindi viene catturata ed analizzata.
	\par L'effettiva misurazione dell'intensità di $M_{xy}$ è tuttavia resa difficile da piccoli fenomeni microscopici. Questi sono responsabili di altri parametri di contrasto intrinseci molto usati %idk here
	oltre a quello della densità protonica già menzionato.
	\subsubsection{Rilassamento dello Spin}
	È importante sottolineare che la frequenza della pulsazione RF deve corrispondere alla frequenza Larmor. Qualunque altra frequenza non sarebbe efficace. Effettivamente, l'assorbimento della pulsazione RF da
	parte del tessuto del paziente corrisponde all'energia di assorbimento dall'insieme degli spin. \par Da un punto di vista classico, la pulsazione RF corrisponde ad un campo magnetico più distante, chiamato
	$\boldsymbol{B_1}$, che ruota alla stessa frequenza di \textbf{M}, e quindi capace di trasferire energia ad \textbf{M} e ruotarlo sul piano $xy$. \par Da un punto di vista meccanico-quantistico, portare lo
	spin ad uno stato eccitatorio richiede il trasferimento di una precisa quantità di energia che corrisponde al divario energetico tra i livelli energetici di spin. Questa è esattamente l'energia di un'onda
	elettromagnetica con frequenza Larmor. Il nome \textit{risonanza magnetica nucleare} si riferisce ad un trasferimento energetico verso un sistema attraverso precise oscillazioni periodiche, e in fisica questo
	fenomeno è chiamato \textit{risonanza}. Il sistema che riceve energia è costituito dall'insieme degli spin dei nuclei di idrogeno all'interno del tessuto in fase di studio. \par Quando l'eccitazione dell'impulso RF
	termina, il sistema tende naturalmente il ritorno al suo stato iniziale, attraverso un fenomeno noto come \textit{rilassamento}. \par Da un punto di vista macroscopico, il rilassamento dopo l'impulso RF può
	essere rappresentato come la combinazione fra due componenti di \textbf{M}: il moto di precessione riguardo $\boldsymbol{B_0}$ che genera un movimento a spirale e descritto da $M_{xy}$ (la proiezione
	di \textbf{M} sul piano $xy$), ed il ritorno di \textbf{M} lungo la direzione di $\boldsymbol{B_0}$, descritto da $M_z$ (la componente \textbf{M} lungo l'asse $z$) (Fig. 1.9a, b). Questi due componenti descrivono
	meccanismi di rilassamento microscopici differenti. La componente longitudinale $M_z$ dipende dalle interazioni e dagli scambi di energia fra gli spin e la struttura molecolare (il ``reticolo''). $M_z$ ritorna in
	equilibrio secondo la curva crescente nella Fig. 1.9a, caratterizzato dal parametro $T_1$, noto come \textit{tempo di rilassamento longitudinale}. Il tempo di rilassamento $T_1$ esprime il tempo che serve
	per il recupero del 63\% del valore di magnetizzazione longitudinale $M_z$ prima dell'impulso RF. il tempo di rilassamento $T_1$ dipende dall'intensità del campo magnetico esterno $\boldsymbol{B_0}$ e dal 
	movimento Browniano interno delle molecole. \par Il rilassamento della componente trasversale $M_{xy}$ dipende dalle stesso fenomeno che contribuisce al rilassamento $T_1$ e da altre interazioni atomiche e
	molecolari inclusa l'interazione spin-spin. $M_{xy}$ ritorna in equilibrio secondo la curva decrescente in Fig. 1.9b, caratterizzata dal parametro $T_2$, noto come \textit{tempo di rilassamento trasversale}.
	Il rilassamento $T_2$ esprime il tempo richiesto dalla magnetizzazione trasversale $M_{xy}$ per decadere al 37\% del suo valore iniziale immediatamente dopo la fine dello stato eccitatorio dell'impulso RF.
	Il rilassamento $T_2$ è essenzialmente causato dalla perdita di coerenza di fase degli spin. In seguito alla pulsazione RF 90° gli spin inizieranno a precedere attorno all'asse di $\boldsymbol{B_0}$ tutti con 
	la stessa fase, ma a causa dell'influenza di vicendevoli campi magnetici microscopici , ruotano con una frequenza di risonanza leggermente diversa e tendono ad essere fuori fase l'uno con l'altro, portando 
	così ad una progressiva distruzione del segnale macroscopico nel piano $xy$. Come regola generale, $T_2$ è più basso nei tessuti solidi e più alto nei tessuti liquidi. \par Nel processo di misurazione realistico,
	la perdita di coerenza di fase degli spin, di conseguenza il rilassamento di $M_{xy}$, può essere accelerata dalla presenza di campi magnetici locali, a causa di variazioni delle variazioni delle componenti tissutali
	locali o una disomogeneità del campo magnetico esterno $\boldsymbol{B_0}$. Questo porta ad un decadimento più veloce di $M_{xy}$, con una curva simile al decadimento di $T_2$, ma definito dal parametro
	$T_2*$%whatt 
	, più corto di $T_2$. In conclusione $T_1$, $T_2$ (e parzialmente $T_2*$) sono parametri intrinseci dei tessuti, hanno valori differenti a seconda del tessuto e possono essere alterati dalla patologia e pertanto
	forniscono una base solida per la creazione di immagini di contrasto. \par La tecnica MR è realizzata tramite la gestione degli impulsi RF ad un sistema e impiegando gradienti di campo magnetico, in precisi intervalli
	di tempo chiamati \textit{sequenze}. \par Un'ampia varietà ed un numero in crescita di sequenze sono attualmente in uso, le quali permettono la creazione di immagini con caratteristiche geometriche 
	e contrasti differenti. I principi della creazione ed ottimizzazione di sequenze di impulsi MR vanno oltre l'ambito di questo capitolo, %non saprei
	in ogni caso i principi base delle sequenze MR ed i parametri principali utilizzati dall'operatore durante l'esame verrano espressi insieme alla spiegazione delle due sequenze le quali strutture sono la base di quasi
	tutte le sequenze MR: spin echo e gradient echo (Weishaupt et al. 2008).
	\subsubsection{Sequenze Spin Echo}
	Questa è una delle famiglie di sequenze più utilizzate dato il loro eccellente valore SNR. \par Lo spin echo (Fig. 1.10) parte ad un impulso eccitatorio RF 90°, il quale ruota \textbf{M} di 90°, dall'asse $z$ al piano
	$xy$. Appena l'impulso è  terminato, la componente $M_{xy}$ inizia a decadere con un rilassamento $T_2*$. Questo segnale è chiamato \textit{decadimento libero dell'induzione} (FID -- free induction decay)
	e di nuovo, questo decadimento è più veloce in confronto al decadimento puro di $T_2$. Invece di acquisire il FID, dopo un intervallo di tempo TE/2, viene gestito un secondo impulso RF, chiamato impulso
	RF 180° con direzione, intensità e durata precise, con l'obbiettivo di ruotare gli spin di 180° sul piano $xy$. Di conseguenza, dopo un altro intervallo di tempo TE/2, gli spin che erano fuori fase durante il
	decadimento $T_2*$ saranno di nuovo in fase sul piano $xy$. L'acquisizione del segnale incomincia in questo momento. È importante sottolineare che l'impulso RF 180° è usato per recuperare il cambio di fase
	degli spin dovuto a variazioni magnetiche locali del tessuto, mentre il cambio di fase dalle interazioni spin-spin non viene recuperato. Quindi, il segnale acquisito su TE sarà l'intensità della componente $M_{xy}$
	decaduta secondo il tempo di rilassamento trasversale $T_2$ (non $T_2*$), ed è così chiamato segnale pesatura-$T_2$. \par Dato che il parametro $T_2$ è una caratteristica riferita al tessuto e può variare
	in condizioni patologiche, attraverso la variazione dell'intervallo TE fra l'impulso RF 90° e l'impulso RF 180°, può essere generato un contrasto fra tessuti con tempi di decadimento $T_2$ differenti (Fig. 1.11).
	\par Se la sequenza viene ripetuta più volte, come avviene di solito in ambienti clinici, è così possibile aggiungere l'effetto di un rilassamento $T_1$ al segnale. TR è il tempo che passa fra due ripetizioni di
	sequenza consecutive. \par Di fatto, dopo il primo impulso RF 90°, mentre $M_{xy}$ si rilassa ed è parzialmente riportato sul piano $xy$ dall'impulso 180°, anche la componente $M_z$ si rilassa e ritorna
	al valore iniziale secondo il rilassamento $T_1$. Quando viene dato l'impulso RF 90° seguente dopo la ripetizione di sequenze spin echo precedente, al tempo TR dal precedente impulso 90°, l'intensità di 
	$M_z$ non sarà necessariamente il massimo valore $M_0$ (punto di partenza), ma avrà il valore raggiunto all'intervallo TR dall'inizio del recupero di $T_1$. \par Se TR ha un valore approssimativamente
	più basso del valore di $T_1$ del tessuto considerato moltiplicato per 3, la componente $M_z$ non avrà abbastanza tempo per ritornare al suo valore massimo; Perciò il segnale acquisito dopo la seconda
	ripetizione spin echo dipenderà da $T_1$. Questo permette la generazione di un contrasto basato sul tempo di rilassamento $T_1$ (Fig. 1.12). \par Fondamentalmente, la pesatura $T_2$ del segnale 
	dipende dal parametro TE il quale è il tempo che intercorre fra la rotazione della magnetizzazione nel piano $xy$ e la sua acquisizione dopo il riorientamento. La pesatura $T_1$ dipende da TR, il tempo fra
	una ripetizione di una sequenza completa e la prossima; Quindi, è il tempo rimasto per il sistema per riempire di nuovo il valore iniziale della componente $M_z$. \par Dal momento che i tessuti del paziente
	hanno valori $T_1$ e $T_2$ differenti, e condizioni patologiche spesso modificano i tempi di rilassamento dei tessuti, allora variando in modo adatto TE e TR è possibile ottenere immagini anatomiche e 
	diagnostiche con contrasti differenti, con pesatura $T_1$ o pesatura $T_2$, in base alla domanda diagnostica. Inoltre, i valori TE e TR possono essere selezionati per ridurre entrambi gli effetti differenti
	di rilassamento$T_1$ e $T_2$ fra tessuti, ottenendo così immagini a densità protonica (Figure 1.13a--e e 1.14a--d).
	\subsubsection{Sequenze Gradient Echo}
	La seconda famiglia fondamentale di sequenze MR è basata sui gradient echo. \par In confronto alle spin echo, le sequenze gradient echo esprimono alcune differenze. Dopo l'impulso RF iniziale che sposta
	\textbf{M} dall'asse $z$ verso il piano $xy$, il riorientamento della componente $M_{xy}$ prima dell'acquisizione è ottenuto attraverso l'uso di gradienti di campo magnetico invece che impulsi RF. I 
	gradienti di campo magnetico sono di fatto una componente essenziale di una sequenza MR e sono inoltre coinvolti nell'acquisizione delle immagini. Questi sono campi magnetici esterni aggiuntivi,
	adeguatamente alterati nelle caratteristiche di spazio e tempo, generati da specifiche bobine le quali sono posizionate attorno al magnete principale. Durante le sequenze, questi campi magnetici
	aggiuntivi sono sovrapposti sul campo magnetico esterno costante $\boldsymbol{B_0}$. Nei gradient echo, l'inserzione di un campo gradiente appropriato dopo il primo impulso RF, insieme ad un 
	gradiente inverso prima del segnale di acquisizione, permette il riorientamento della magnetizzazione sul piano $xy$. Tuttavia, in confronto allo spin echo, il segnale che è perduto a causa di variazioni locali
	del campo magnetico non sarà recuperato ed il segnale acquisito sarà pesatura-$T_2*$. Questo rende le immagini gradient echo molto più sensibili ai cambiamenti nella predisposizione magnetica dei tessuti,
	e potrebbe quindi generare artefatti in alcuni casi. D'altra parte, alcune caratteristiche possono fornire informazioni riguardo depositi con proprietà magnetiche speciali e di rilevanza clinica , come calio o 
	emosiderina. \par Il vantaggio più notevole dell'imaging gradient echo è una generazione di immagini più rapida a differenza di spin echo. Infatti, gradient echo usa un FA minore di 90°, permettendo a
	TR di decrescere in modo da ottenere un'acquisizione più rapida di immagine al costo di un segnale decrementato ed una qualità d'immagine inferiore (Fig. 1.15). La Figura 1.16 riassume i tratti distintivi del
	segnale dei fluidi, tessuti adiposi e osso corticale, nel caso di sequenze spin echo.
	\subsubsection{Ricostruzione dell'Immagine}
	Come in ogni tecnica di imaging tomografica, l'imaging MR rappresenta volumi attraverso immagini 2D. Un voxel è un'unità elementare del volume di ogni regione anatomica sotto analisi, mentre il pixel è
	l'unità elementare dell'immagine MR che rappresenta quella stessa regione anatomica. Un voxel ha tre dimensioni, mentre il pixel ha due dimensioni. Le regioni anatomiche possono essere divise in delle 
	sezioni, ognuna contenente uno strato di voxels. Ogni sezione sarà rappresentata in una singola immagine bidimensionale e perciò il pixel può essere considerato come la rappresentazione bidimensionale
	di un voxel (Fig. 1.17). La profondità del voxel corrisponderà quindi allo spessore della sezione, mentre il piano della sezione è una matrice composta da righe e colonne di singoli pixel.
	\par La codifica spaziale nella MR è il processo per cui la posizione di una sorgente di segnale di un singolo voxel è identificato nello spazio. Come già menzionato, secondo l'equazione di Larmor, la
	frequenza di precessione $\omega_0$ dello spin individuale è direttamente proporzionale all'intensità del campo magnetico statico $\boldsymbol{B_0}$, attraverso il rapporto giromagnetico $\gamma$,
	una costante fisica tipica dell'elemento chimico a cui appartiene lo spin. Se l'intensità del principale campo magnetico $\boldsymbol{B_0}$ non è più costante ma varia progressivamente nello spazio
	secondo un gradiente lineare \textit{G} anche la frequenza di precessione dello spin individuale varia linearmente nello spazio, permettendoci di identificare la posizione degli spin lungo la direzione del
	gradiente, basata sulla loro frequenza di risonanza (Fig. 1.18). Il sistema di codifica spaziale per mezzo di gradienti perciò modula l'intensità del campo magnetico principale nello spazio attraverso 
	bobine a gradiente che sono ulteriori conduttori di carica contenuti nel sostegno. La loro attivazione è responsabile dell'intenso rumore durante l'esame MR. \par Nel caso di sequenze 2D, così è come si
	formano le immagini. La selezione della sezione è ottenuta tramite l'applicazione di gradiente di campo magnetico lineare lungo un asse, per esempio, sull'asse $z$. Quindi, lungo l'asse $z$, le frequenze
	Larmor cambieranno gradualmente e per ogni sezione ci sarà una specifica risonanza. Ancora una volta, i protoni vengono eccitati solo da un impulso RF con una frequenza uguale alla loro frequenza
	Larmor. Un impulso RF che corrisponde alla frequenza Larmor dei protoni localizzati nella sezione desiderata produrranno una reazione solo nei protoni di quella sezione e le sezioni rimanenti del corpo
	anatomico saranno escluse dall'eccitamento. \par Dopo la scelta della sezione, le coordinate dei voxel sul piano $xy$ sono da identificare. La codifica su uno dei due assi, per esempio sull'asse $y$,
	avviene attraverso l'azione di un gradiente di campo magnetico lungo l'asse su cui è situato in quel dato momento. Questo gradiente indurrà nuovamente una variazione della frequenza di risonanza 
	dello spin lungo la direzione del gradiente, col risultato di un tasso di precessione differente. Quando il gradiente è spento, la frequenza di precessione sarà di nuovo la stessa per tutti gli spin, ma
	al contempo gli spin hanno accumulato una differenza di fase, a causa della loro posizione lungo l'asse $y$. Questa è chiamata codifica di fase. La codifica su linee differenti richiede una ripetizione con
	intensità di gradiente di fase differenti. \par Infine, la codifica lungo l'asse $x$ è ottenuta tramite l'applicazione di gradiente lungo l'asse $x$ durante il campionamento echo. Questo significa ottenere
	simultaneamente tutte le frequenze che corrispondono alle posizioni degli spin lungo una linea parallela all'asse $x$. Perciò, nella sezione prescelta, le coordinate $xy$ di un voxel singolo saranno 
	identificate basandosi sulla frequenza (coordinata $x$) e della fase (coordinata $y$) del segnale acquisito. \par In ultima analisi dopo la cattura di tutti i dati da ogni sezione, il risultato sarà un ammasso
	di dati corrispondente all'intero volume di interesse. Questo set ordinato di segnali fornisce il \textit{k-space}, noto come \textit{spazio dei dati grezzi}. \par il \textit{k-space} non è uno spazio fisico
	ma una matrice bidimensionale con dati numerici, ma i pixel non corrispondono direttamente l'uno con l'altro. Il \textit{k-space} contiene molte informazioni riguardo allo spazio reale che rappresenta, ma in 
	una modalità in codice. I coefficienti $k$ più alti (corrispondenti alle frequenze di segnale più alte), i quali contribuiscono ai dettagli anatomici dell'imagine, sono posizionati nella parte periferica del
	$k-space$. I coefficienti $k$ più bassi (corrispondenti alle frequenze di segnale più basse), i quali contribuiscono al contrasto dell'immagine, sono posizionati nell'area centrale (Fig. 1.19). Quando tutti
	i punti del \textit{k-space} vengono catturati alla fine della scansione, i dati possono essere ricostruiti per generare l'immagine. Per passare dal \textit{k-space} (dati grezzi) all'immagine anatomica,
	la quale è costituita da pixel con intensità differente in relazione alla posizione spaziale del voxel anatomico, è utilizzata un'operazione matematica: La traformata di Fourier. 
	\par La risoluzione spaziale dell'immagine anatomica finale è data dallo spessore della sezione e, sul piano della sezione, dalla dimensione del campo visivo (FOV -- \textit{field of view}), dal numero di 
	codifica di fasi (righe del \textit{k-space}) e dalle codifiche di frequenza (numero di punti acquisiti lungo una linea del \textit{k-space}). \par È importante sottolineare che l'SNR dell'immagine finale 
	dipende oltretutto sulla quantità del tessuto contenuto nel voxel dal quale il segnale viene captato, e le sue dimensioni sono definite da una risoluzione spaziale scelta durante la prescrizione della
	sequenza. Percò, quando tutti gli altri parametri sono uguali, l'aumento della risoluzione spaziale è di fatto limitata dal segnale MR intrinsecamente debole del tessuto. La ripetizione della sequenza
	potrebbe aiutare all'accrescimento del parametro \textit{signal-to-noise}; tuttavia, questo può essere limitato dal movimento del corpo in esame e della durata dell'indagine.
	\subsubsection{Acquisizione 2D e 3D}
	L'acquisizione bidimensionale può essere a sezione singola o sezione multipla. Nella modalità a sezione singola, le sezioni sono completamente acquisite in successione. Nella tecnica a sezione
	multipla invece, dopo l'acquisizione di una porzione di una sezione, il sistema passa alla sezione seguente per poi tornare al completamento della prima sezione solamente dopo. Questo permette 
	di accorciare i tempi di acquisizione dato che, durante l'attesa fra l'eccitazione di una sezione e l'altra (TR), è possibile eccitare ed acquisire linee di \textit{k-space} di altre sezioni.
	\par Le tecniche di acquisizione 3D hanno un SNR alto e sono sempre di più usate per la scansione di volumi con un'alta risoluzione spaziale in tutte le direzioni. I dati possono essere acquisiti e 
	ricostruiti con qualunque orientamento nello spazio, permettendo la visione del volume di interesse su ogni piano, con voxel isotropici e con dettagli notevoli. Le acquisizioni 3D sono effettuate
	tramite l'uso di un gradiente di fase aggiuntivo che permette la codifica di fase nella direzione della sezione. \par Dal momento che le direzioni spaziali sono codificate da gradienti di campo magnetico
	creati da bobine inserite all'interno dello scanner, le acquisizioni MR, a differenza di altre tecniche di imaging, possono essere direttamente effettuate lungo piani obliqui. Infatti, un gradiente di campo
	in una qualunque direzione obliqua può essere ottenuto dalla somma delle tre componenti, generate dalle bobine gradiente lungo gli assi Cartesiani definiti dalla geometria dello scanner. Con
	l'acquisizione 3D, c'è inoltre la possibilità di riformattare i dati dopo l'acquisizione, lungo piani diversi dal piano di acquisizione. La buona qualità delle ricostruzioni multiplanari, le quali sono ottenute
	su piani differenti dal piano di acquisizione, è ancora dovuto all'uso di voxel isotropici (uno con tre lati uguali).
	\subsubsection{Tecniche di Acquisizione Rapida}
	Nel corso degli anni, i sistemi clinici MR hanno fornito un numero in aumento di sequenze e tecniche, molte delle quali mirano alla riduzione dei tempi di acquisizione attraverso modalità differenti
	di campionamento del \textit{k-space}. Di seguito vi è una breve descrizione delle più importanti, con riferimenti ad altri testi per un'analisi più approfondita (Weishaput et. al 2008; Elmaoĝlu e
	Çelik 2012). \par Turbo spin echo (TSE), anche noto come spin echo rapido (FSE -- \textit{fast spin echo}), è una sequenza spin echo nella quale avvengono impulsi molteplici in seguito il primo
	impulso 180°, ognuno dei quali genera un eco che è acquisito ed una ripetizione di sequenza riempie più righe del \textit{k-space}. La codifica di fase è ottenuta dal cambio su un gradiente di fase,
	prima di ogni impulso 180° e dopo l'acquisizione dell'eco, ma con polarità inversa, per cancellare il cambio di fase e permettere una nuova codifica di fase prima del prossimo eco. I parametri
	addizionali per questa sequenza sono eco train length (ETL, anche chiamato fattore turbo TF), che sono il numero di eco generati durante un TR, ed echo spacing (ESP), che è il tempo fra un 
	eco e quello seguente, durante la ripetizione medesima della sequenza. In generale, l'acquisizione di tutto il \textit{k-space} richiederà meno ripetizioni di sequenza e quindi una notevole 
	riduzione dei tempi di acquisizione. \par Alcune tecniche che sono attualmente in uso per ridurre i tempi di acquisizione consistono nel ridurre il numero di righe del \textit{k-space}, addirittura
	al costo di un FOV rettangolare, con un pixel non più quadrato ma rettangolare, o applicando una riduzione dell'SNR, a seconda della tecnica. Il risultato sarà in ogni caso in linea alla qualità di 
	immagine che è richiesta dal bisogno clinico. \par Un'altra strategia di acquisizione rapida del \textit{k-space} è l'EPI (echo planar imaging). In questo caso, la frequenza del gradiente alterna 
	velocemente fra valori positivi e negativi. Ogni valore è associato ad una lettura d'eco e le linee del \textit{k-space} sono riempite in direzioni differenti in modo alternato. Con questa tecnica,
	è inoltre possibile l'acquisizione dell'intero \textit{k-space} dopo un singolo impulso di eccitamento.
	\subsubsection{Soppressione del Fat Signal}
	I segnali nell'imaging MR sono principalmente generati da protoni di acqua contenuti nei tessuti. Tuttavia in diverse regioni anatomiche i segnali che provengono dai protoni del grasso avranno
	un impatto eccessivo, con i pixel del tessuto grasso che appaiono in modo iperintenso. A causa di ciò, la valutazione clinica diventa più difficoltosa e possono essere generati degli artefatti.
	\par Con l'obbiettivo della riduzione del fat signal, vengono utilizzate strategie differenti e sono indicate come teniche \textit{fat suppression}. Le sequenze cliniche spesso contengono uno
	step di preparazione a contrasto preliminare per ridurre il fat signal. \par Una sequenza solitamente usata per il fat suppression è la STIR (short tau inversion recovery), la quale è una versione
	più generale della sequenza IR (inversion recovery). \par Nella IR, prima dell'impulso 90° che sposta la magnetizzazione sul piano $xy$, è dato un impulso 180° che inverte la magnetizzazione
	di 180° rispetto all'asse $z$. Alla fine dell'impulso 180°, la componente $M_z$ tenderà all'equilibrio secondo il rilassamento $T_1$, come già descritto. Dopo uno specifico intervallo, noto come
	TI (inversion time), l'impulso 90° sarà generato all'avvio della sequenza descritta in precedenza (spin echo o gradient echo). L'utilità dell'impulso di inversione di 180° è dovuto al fatto che, 
	quando ritorna all'equilibrio, segnali provenienti da tessuti con differenti $T_1$ attraverseranno il valore zero in tempi differenti. Attraverso un appropriata scelta del valore TI, è possibile
	eliminare il segnale di un tessuto distribuendo l'impulso di eccitamento 90° quando la componente $M_z$ di quel tessuto è nulla, così il segnale sul piano $xy$, $M_{xy}$, non conterrà
	la componente che viene eliminata. \par Nel caso delle sequenze STIR, il parametro di tempo TI è scelto in modo da eliminare il fat signal (Fig. 1.20). Un'altra tecnica (tecnica di saturazione
	del grasso) per ridurre il fat signal è basata su frequenze di risonanza differenti fra protoni d'acqua e protoni di grasso. Precisamente, come menzionato in precedenza, il campo magnetico
	locale percepito dagli spin dipende ulteriormente dalla struttra molecolare che li circonda, questo causa la frequenza di risonanza dei protoni che appartengono alle molecole di grasso è 
	leggermente diversa da quella dei protoni legati alle molecole d'acqua, che sono strutturalmente differenti. La differenza di frequenza aumenta proporzionalmente al campo magnetico
	esterno $\boldsymbol{B_0}$. Quindi, le tecniche di soppressione del fat signal le quali sono note come tecniche a frequenza selettiva, adoperano l'uso di un impulso di eccitamento iniziale 90°
	solo sulla banda di frequenza del grasso, seguito dal cambio di fase nel piano $xy$, possibilmente accelerato da un gradiente di deterioramento. Dopodichè la componente $M_z$ disponibile
	per l'acquisizione rimane solo quello per la componente dell'acqua che sarà il segnale primario dell'immagine generata con la seguente parte della sequenza. \par Un'altra importante tipologia 
	di separazione dell'acqua è chiamata \textit{chemical shift imaging}. In questo caso, è impiegato il cambio fra il segnale d'acqua ed il fat signal. Dopo l'impulso 90°, il quale ruota
	longitudinalmente la magnetizzazione nel piano $xy$, la componente magnetizzazione trasversale per grasso e acqua inizierà il suo moto di precessione con velocità differenti, e raggiungerà
	il massimo del cambio di fase di 180° dopo un certo intervallo di tempo che viene scelto come TE (tempo di eco). Questo permetterà la creazione di un'immagine ``fuori-fase''. Dopo lo stesso
	intervallo di tempo, i due componenti ritorneranno in fase, ed un'immagine ``in-fase'' della somma di segnali d'acqua e grasso può essere acquisita. Fra questi tipi di sequenze, la tecnica Dixon
	è ampiamente utilizzata. Attraverso la combinazione delle immagini ``fuori-fase'' e ``in-fase'' pixe-su-pixel, solo immagini del segnale d'acqua e del segnale grasso sono ottenuto.
	\par Le tecniche di soppressione del grasso sono estremamente utili nella diagnostica. Esse possono evidenziare piccole aree di edema, così come l'involuzione di un muscolo e tessuti critici.
	La sostituzione di midollo osseo con infiltrazioni edematose possono essere individuate con la soppressione del fat signal. Questo può verificarsi nei processi di sovraccarico funzionale il quale 
	risultato è il rimodellamento condilare regressivo, con l'edema del midollo osseo come uno dei primi segni dell'artrite dell'ATM (Fig. 1.21a). Con queste sequenze, le caratteristiche dei muscoli
	sopramandibolari possono essere fattori di studio (Fig.1.21b). \par Infine, le cosiddette \textit{bande di saturazione} sono comunemente usate nella pratica clinica (Fig. 1.22). L'operatore 
	disegna queste bande su un'immagine in corrispondenza delle regioni da dove il segnale dev'essere soppresso. L'obiettivo è generalmente l'eliminazione dei segnali di grasso notevoli e locali
	da una regione vicino all'area di interesse, evitando così errori. In questo caso, gli impulsi RF 90° sono utilizzati in combinazione con gradienti di campo, convenientemente posizionati nello 
	spazio così che le aree possano essere selezionate dove il segnale dev'essere saturato. Il segnale selezionato sarà ora nel piano $xy$ e sarà perciò non più disponibile lungo l'asse $z$, nel
	momento in cui la sequenza di imaging partirà.
	\subsubsection{Parametri e Ottimizzazione delle Sequenze}
	La MRI si differenzia dalle scansioni CT e PET in quanto contrasto e risoluzione dipendono da molteplici parametri differenti -- alcuni intrinseci, altri estrinseci -- che si influenzano reciprocamente
	in modo variabile, a seconda dell'anatomia e della sequenza scelta. Durante l'acquisizione, tutte le sequenze MR implicano una selezione di un certo numero di parametri estrinseci, in base al tipo 
	di tessuto, il volume di interesse, SNR, riduzione di errori, campo magnetico, bobina da utilizzare, durata della procedura, contrasto desiderato e risoluzione (Manoliu et al. 2016). In aggiunta,
	all'interno di un certo sistema o di una regione, i parametri sono intrecciati: cambiarne uno significa cambiarne altri, e la stessa risoluzione e contrasto può essere ottenuta con sequenze differenti
	o con parametri differenti della stessa sequenza. Questa complessità richiede una costante ottimizzazione che prende forma in tre livelli diversi.
	
	
	
	




















\end{document}